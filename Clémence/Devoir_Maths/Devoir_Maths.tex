\documentclass{article}
%% to use a package, just delete the % character to enable the package in this document, only on the lines where there is a single % character (otherwise it won't compile)
%% to understand the utility of the package, please read the comment between \begin{comment} and \end{comment} right below the package you need
\usepackage[a4paper, total={6in, 8in}]{geometry} % for european user

%% for multiple lines comment in LaTeX
\usepackage{verbatim} % (won't compile if this line is deleted)
%% usage :
\begin{comment}
	Your multiple lines comment
\end{comment}


%% to include images
%\usepackage{graphicx}
%% usage :
\begin{comment}
	\begin{figure}
		\centering (not necessary but it's prettier)
		\includegraphics[scale=(1 by default)]{path to the image/imageName.extension}
		\if you want to put another image side to side, just add a line like the previous
		\caption{picture(s) in a nutshell} (not necessary but it's useful) (won't compile if missing)
	\end{figure}
\end{comment}


%% pour insérer des liens
%\usepackage{hyperref}
%% usage :
\begin{comment}
	\href{the complete link}{the display text}
	or if you want to show the complete link, just use 
	\url{the complete link}
\end{comment}


%% to display code easily (insert --shell-escape option when compile, otherwise, it won't compile)
%\usepackage{mint}
%% usage :
\begin{comment}
	- first, you have to have python3-pygments installed on your computer
	\begin{minted}{<language you choose>
		Your code
	\end{minted}
\end{comment}

%% to display code (for more informations, please see : https://texdoc.org/serve/listings.pdf/0)
%\usepackage{listings} 
%% usage :
\begin{comment}
	- listings package let you highly customize how you code will be displayed on the document
	- for simple display just do this :
	\lstdefinestyle{mystyle}{
		Your customizations
	}
	\begin{lstlisting}[language=<the language you'll use)
		Your code
	\end{lstlisting}
\end{comment}

%% to use maths and other sciences symbols (cf https://www.cmor-faculty.rice.edu/~heinken/latex/symbols.pdf) (
\usepackage{amssymb,amsmath,amsfonts,extarrows}
%% usage : just type the backslash chararacter and the name of the symbol you want, depending of document on the the previous link

%% Have to find the goal of it
\usepackage{soul}
\let\oldemptyset\emptyset
\usepackage[T1]{fontenc}

%% won't compile if at least one of these three next line is deleted
%% but you can add nothing between the brackets to leave it blank
\author{}
\date{}
\title{}

\begin{document}
%% not necessary
%% compile twice the first time to display table of content
%\tableofcontents
%\newpage

\section{Exercice 1}
Sur une étagère se trouvent 7 livres différents : 4 livres de mathématiques, 1 livre de philosophie et 2 livres de cuisine.\newline
On choisit 2 livres de l'étagère.
\begin{enumerate}
	\item Combien de choix y a-t-il ?\newline
	L'ordre n'est pas important donc il y a 2 parmi 7 permutations possibles : $\frac{7!}{2!(7-2!)} = 21$ (Je lui laisse le raisonnement et la rédaction à faire)
	\item Combien y a-t-il de choix avec 1 livre de mathématiques et 1 autre livre, choisi en dehors des Mathématiques ?\newline
	Pour chaque livre de maths, il n'y a que 3 possibilités : philo/cuisine A, philo/cuisine B et cuisine A/cuisine B. Il y a 4 livres de maths, il y a donc 4$\times$3 = 12 possibilités.
\end{enumerate}


\section{Exercice 2}
\begin{enumerate}
    \item 
	\begin{flalign*}
		\lim_{n\to +\infty} \frac{2n+1}{n^{4}-3n^{2}+1} & = \lim_{n\to +\infty} \frac{2n}{2n}\frac{1+\frac{1}{2n}}{\frac{n^{3}}{2}-\frac{3n}{2}+\frac{1}{2n}} &\\
														& = \lim_{n\to +\infty} \frac{1+\frac{1}{2n}}{\frac{n^{3}}{2}-\frac{3n}{2}+\frac{1}{2n}} & \\
														& \text{or } \lim_{n\to +\infty}1+\frac{1}{2n}=0 \text{ et } \lim_{n\to +\infty}{\frac{n^{3}}{2}-\frac{3n}{2}+\frac{1}{2n}} = \lim_{n\to +\infty}\frac{n^{3}}{2} = +\infty &\\
														& \Longrightarrow \text{ par quotient de limite (pas sûr de cette formulation) : }\lim_{n\to +\infty} \frac{2n+1}{n^{4}-3n^{2}+1} = 0 &\\
	\end{flalign*}

	\item
	\begin{flalign*}
		\lim_{n\to +\infty} \left(3+\frac{1}{n}\right)^{n} & : \forall n\in\mathbb{N}, 3<3+\frac{1}{n}\leq 4 \text{ (je ne sais pas si elle doit expliquer pourquoi plus en détail) (la comparaison avec le 4 est pas importante mais avec le 3 oui) } &\\
														   & \text{ or } \lim_{n\to +\infty} 3^{n} = +\infty &\\
														   & \Longrightarrow \text{ par théorème de comparaison : } \lim_{n\to +\infty} \left(3+\frac{1}{n}\right)^{n} = +\infty &\\
	\end{flalign*}
\end{enumerate}

\newpage
\section{Exercice 3}
\begin{enumerate}
	\item Montrer par récurrence que pour tout n de $\mathbb{N}$, n$<$50.\newline
	Je la laisse faire la partie rédaction.
		\begin{itemize}
		\item Initialisation (n=0) : $u_{0} = 5 < 50$
		\item Récurrence : on suppose la propriété vrai à un rang k, k$\in\mathbb{N}$.
			\begin{flalign*}
				u_{k} < 50 & \Longleftrightarrow 0.8u_{k} < 40 \text{ } (40=50\times0.8) &\\
					   	   & \Longleftrightarrow 0.8u_{k}+10 < 50 &\\
					       & \Longleftrightarrow u_{k+1} <50
			\end{flalign*}
		\end{itemize}
	\item En déduire que la suite ($u_{n}$) est strictement croissante.\newline
		(C'est toujours la question avec laquelle j'ai du mal, je vais essayer de retrouver comme avoir la réponse.)
	\item En déduire que la suite ($u_{n}$) est convergente .\newline
		La suite est strictement croissante (question précédente) et majorée (question 1) donc ($u_{n}$) est convergente.
\end{enumerate}

\newpage
\section{Exercice 4}
\begin{enumerate}
	\item Calculer f'(x) pour tout x de $\mathbb{R}$.\newline
		\begin{flalign*}
			f'(x) & = \left(e^{x}-x\right)' &\\
				  & = \left(e^{x}\right)' - (x') \text{ en dérivant par rapport à x on obtient : } &\\
				  & = e^{x} - 1 &\\
		\end{flalign*}
	\item Dresser le tableau de variation de f sur $\mathbb{R}$ et en déduire que pour tout réel x : f(x)$\geq$1.\newline
		Pour dresser le tableau de variation de f sur $\mathbb{R}$, on passe par le tableau de signe de f' : $e^{x}-1<0 \Longleftrightarrow e^{x}<1 \Longleftrightarrow x<0$.
		f'(x) est donc négative pour tout x$<$0 et positive pour tout x$>$0.\newline
		En résumé : si x<0, f'(x)<0 donc f(x) décroissante et si x>0, f'(x)>0 donc f(x) croissante.
	\item Soit ($u_{n}$) la suite définie par : $u_{0} = 0$ et $u_{n+1} = e^{u_{n}}$
		\begin{itemize}
			\item (a)Montrer par récurrence que pour tout n de $\mathbb{N}, u_{n}\geq n$.\newline
				Je la laisse se charger de la rédaction complète.
				\begin{itemize}
					\item Initialisation (n=0) : $u_{0} = 0 \Longleftrightarrow u_{0}\geq 0$
					\item Hérédité : on suppose la propriété vraie à un rang réel k.
					\begin{flalign*}
						u_{k} \geq k & \Longleftrightarrow e^{u^{k}} \geq e^{k} \text{ car la fonction exponentielle est strictement croissante sur }\mathbb{R} &\\
									 & \text{or, pour tout entier naturel k, }e^{k}\geq k+1 \text{ (cas extrême k=0 et exp(0)=1=0+1) donc } &\\
									 & \Longrightarrow e^{u_{k}}\geq k+1 &\\
									 & u_{k+1} \geq k+1 \\
					\end{flalign*}
				\end{itemize}
			\item (b)En déduire que la suite ($u_{n}$) est divergente.\newline
				On a montré à la question 3.a que pour tout entier naturel n, $(u_{n}) \geq n$, or n diverge en +$\infty$. Par thèorème de comparaison, la suite ($u_{n}$) diverge.
		\end{itemize}
\end{enumerate}

\end{document}
