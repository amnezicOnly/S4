\documentclass{article}
\usepackage[a4paper, total={6in, 8in}]{geometry}  
\usepackage{graphicx} % pour insérer des images

% pour utiliser des notations scientifiques
\usepackage{amssymb}
\usepackage{amsmath}
\usepackage{amsfonts}

\usepackage{extarrows} % pour afficher les flèches de logiques (implique, équivalent à ,etc...)
\usepackage{soul} % utilité à troiver
\let\oldemptyset\emptyset
\usepackage[T1]{fontenc}

\author{Amnézic}
\date{}
\title{Aide Vim}

\begin{document}
\maketitle
\newpage
\tableofcontents
\newpage
\section{Introduction}
\subsection{Présentation}
Pour trouver de l'aide sur une commande ou un sujet, il suffit d'être en mode normal et de taper: :[commande] ou :[subject]. Pour l'utilisation d'une commande dans un certain mode, il suffira de taper:[mode]\_[commande ou sujet]. Les différents modes sont :
\begin{itemize}
    \item i (insert mode)
    \item v (visual mode)
    \item c (command line mode)
\end{itemize}
Toutes les comandes seront présentées dans un tabularau à deux colonnes, la première avec la commande pour une seule itération et la seconde colonne avec la commande pour \textit{n} itérations.
\subsection{Commandes de base}
Se déplacer (les flèches directionnelles fonctionnent aussi) (dans le sens horaire en partant d'en haut): k, l, j et h.\newline
Pour insérer du texte : i[texte]

\section{Modifier le document}

\subsection{Se déplacer efficacement}
\begin{center}
\newline\begin{tabular}{|l|l|l|}
    \hline
    Début des mots suivants & w & \textit{n}w \\
    \hline
    Fin des mots suivants & e & \textit{n}ge \\
    \hline
    Début des mots précédents & b & \textit{n}b \\
    \hline
    Fin des mots précédents & ge & \textit{n}ge \\
    \hline
    Aller au début de la ligne & 0 & \\
    \hline
    Aller à la fin de la ligne & \$ & \\
    \hline
    Aller au prochain [x] & f[x] & \textit{n}f[x] \\
    \hline
    Aller au précédent [x] & F[x] & \textit{n}F[x] \\
    \hline
    Aller au caractère ouvrant/fermant correspondant & \% & \\
    \hline
    Aller en haut du document & H & \\
    \hline
    Aller au milieu du document & M & \\
    \hline
    Aller en bas du document & L & \\
    \hline
    Ajouter une ligne au-dessus & maj-o & \textit{n}maj-o \\
    \hline
    Ajouter une ligne en-dessous & o & \textit{n}o\\
    \hline
    Aller à la ligne & \textit{n} & :\textit{n} \\
    \hline
    Se déplacer à la prochaine occurence de [texte] & /[texte] & \\
    Aller au suivant & n & \\
    \hline
    A revoir & mots qui  se finissent & par [texte] \\
    \hline
\end{tabular}
\end{center}




\subsection{Insérer du texte}





\subsection{Supprimer}
\begin{center}
\newline\begin{tabular}{|l|l|l|}
    \hline
    Des caractères & x & \text{n}x \\
    \hline
    Des mots (se place au début du mot restant) & dw & d\textit{n}w\\
    \hline
    Des mots (se place avant le mot restant) & de & d\textit{n}e \\
    \hline
    Supprimer le mot courant & daw & \textit{n}daw \\
    \hline
    Des lignes depuis la courante & dd & \text{n}dd \\
    \hline
    Supprimer le reste de la ligne & D & \\
    \hline
    Supprimer la ligne jusq'au curseur (non-inclus) & d0 & \\
    \hline
    Supprimer le contenu jusqu'à la ligne courante & dgg & \\
    Supprimer le contenu jusqu'a la fin du fichier & dG & \\
    \hline
    Remplacer un caractère par [x] & rx & r\textit{n}x \\
    \hline
    Changer du texte & cw[texte] & c\textit{n}w[texte] \\
    \hline
    Changer le reste de la ligne & C & \\
    \hline
    Changer toute la ligne & S & \\
    \hline
\end{tabular}
\end{center}



\subsection{Séléctionner du texte}
Pour aider à voir ce que vous faites, vous pouvez vous mettre en mode visuel, simplement en cliquant sur v à partir du mode normal.
\begin{center}
\newline\begin{tabular}{|l|l|l|}
    \hline
    Séléctionner \textit{n} lignes & Vj & V\textit{n}j \\
    \hline
    Copier du texte (avec espace final) & yw & y\textit{n}w \\
    Copier du texte (sans espace final) & ye & y\textit{n}e \\
    Copier le reste de la ligne courante & y\$ & \\
    Copier toute la ligne & Y & \\
    \hline
    Coller du texte (du texte est supposé copié) & p & \\
    \hline
\end{tabular}
\end{center}


\section{Commandes sur le fichier}
\begin{center}
\newline\begin{tabular}{|l l|}
    \hline
    Sauvegarder le fichier & :w \\
    \hline
    Sortir du fichier sans sauvegarder & :q! \\
    \hline
    Sortir du fichier et sauvegarder & ZZ \\
    \hline
    Laisser le fichier en suspend & Ctrl-z (puis fg dans la CLI pour revenir dessus) \\
    \hline
    Ouvrir les fichiers les uns derrière les autres & vim file1 ... fileN \\
    Aller au fichier suivant & :(w pour sauvegarder)next(! ne pas enregistrer)\\
    Aller au n-ième fichier suivant & :(w pour sauvegarder)\textit{n}next(! ne pas enregistrer) \\
    Aller au fichier précédent & :(w pour sauvegarder)previous(! ne pas enregistrer)\\
    Aller au n-ième fichier précédent & :(w pour sauvegarder)\textit{n}previous(! ne pas enregistrer) \\
    Aller au premier fichier & :(w pour sauvegarder)first(! ne pas enregistrer)\\
    Aller au dernier fichier & :(w pour sauvegarder)last(! ne pas enregistrer)\\
    Redéfinir les fichiers ouvert & :args newFile1 ... newFileN \\
    \hline
    Ouvrir les fichiers côte à côte & vim -O file1 ... fileN  \\
    Se déplacer au fichier à gauche & ctrl-maj-w h\\
    Se déplacer au fichier à droite & ctrl-maj-w l\\
    (Vim) Ouvrir un nouveau fichier à côté du premier & :vsplit file1 ... fileN \\
    \hline
    Ouvrir les fichiers empilés & vim -o file1 ... fileN \\
    Se déplacer au fichier au-dessus & ctrl-maj-w k\\
    Se déplacer au fichier en-dessous & ctrl-maj-w j\\
    Ouvrir un nouveau fichier en-dessous du premier & :split file1 ... fileN \\
    \hline
    Quitter tous les fichiers & qall! \\
    Sauvegarder tous les fichiers & wall \\ 
    \hline
    Copier coller du texte dans 2 fichiers différents & :edit file1 \\
     & sélection du texte et copiage \\
     & :edit file2 \\
     & p \\
    \hline
    Revoir page 74 \\
    \hline
    Voir le contenu d'un fichier & vim -R file \\
    \hline
    Passer le fichier en read-only (permanent?) & vim -M file \\
    \hline
    Créer un nouveau fichier à partir d'un fichier déjà existant & aller dans le fichier déjà existant \\
     & :edit existingFile \\
     & (travail sur le nouveau fichier) \\
     & :saveas newFile \\
    \hline
    Changer le nom du fichier courant & :file newFileName \\
    \hline
\end{tabular}
\end{center}


\section{Autres}
\subsection{Les dossiers}
Pour "raccourcir" le fichier, il est possible de demander à Vim de créer des dossiers, c'est-à-dire réduire ce qui peut être réduit. Donc si vous avez un fichier avec disons 5 fonctions et que chaque fonction fait 20 lignes, de devoir se déplacer parmi les 100 lignes, il sera possible de se déplacer parmi 5 lignes. Pour cela, il suffit de modifier le vimrc.\newline
Il y a plusieurs méthodes de fold :
\begin{itemize}
    \item manual : il faut faire manuellement chaque dossier
    \item indent : crée les dossiers à partir de l'indentation
    \item syntax : crée les dossiers à partir de la syntaxe du langage du fichier
\end{itemize}

\subsection{Recherche}
Pour rechercher un caractère ou une suite de caractères dans le fichier, il suffit de faire (en mode normal) /[texte]. Le texte est en réalité interprété comme une regex. Donc :
\begin{itemize}
    \item /a* : permet de chercher "","a","aa",...
    \item /ab$\backslash$+ : permet de chercher "ab","abb",...
    \item /ab$\backslash$- : permet de chercher "a" ou "ab"
    \item /word1$\backslash$|...$\backslash$|wordN : permet de chercher word1 ou ... ou wordN
    \item /$\backslash$(ab$\backslash$)$\backslash$* ou + ou - : permet de faire les mêmes recherches mais sur des groupes de caractères 
    \item /text : permet de chercher le texte
    \item /text$\backslash$ : permet de chercher uniquement le texte que l'on veut (par exemple, si on cherche $\backslash$i, if et while ne seront pas surlignés)
    \item /(\^\ texte) : permet de chercher tout sauf le texte
\end{itemize}
Pour ceux qui ont des notions de regex : * correspond à *, a$\backslash$+ correspond à a$^{+}$ et a$\backslash$- correspond à a$^{?}$. Les ensembles \textbf{évidents} tels que [a-z],[A-Z] et [0-9] (qui peuvent être combinés d'ailleurs), peuvent aussi être utilisés à la place des caractères ou des mots. D'ailleurs, afin de faciliter la recherche de ces ensembles, il y a les raccourcis suivants (doivent toujours être précédés de / pour la recherche):
\begin{itemize}
    \item $\backslash$d : [0-9]
    \item $\backslash$x : [0-9a-fA-F]
    \item $\backslash$s : caractères espaces
    \item $\backslash$l : [a-z]
    \item $\backslash$u : [A-Z]
    \item remarques : pour trouver le complémentaire de chaque ensemble, il suffit de faire $\backslash$<la lettre en majuscule>. Par exemple, pour chercher tout sauf les nombres de 0 à 9, on fera /$\backslash$D.
\end{itemize}

\end{document}
