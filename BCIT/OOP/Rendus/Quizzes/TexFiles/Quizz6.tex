\documentclass{article}
%% to use a package, just delete the % character to enable the package in this document, only on the lines where there is a single % character (otherwise it won't compile)
%% to understand the utility of the package, please read the comment between \begin{comment} and \end{comment} right below the package you need
\usepackage[a4paper, total={6in, 8in}]{geometry} % for european user

%% for multiple lines comment in LaTeX
\usepackage{verbatim} % (won't compile if this line is deleted)
%% usage :
\begin{comment}
	Your multiple lines comment
\end{comment}


%% to include images
%\usepackage{graphicx}
%% usage :
\begin{comment}
	\begin{figure}
		\centering (not necessary but it's prettier)
		\includegraphics[scale=(1 by default)]{path to the image/imageName.extension}
		\if you want to put another image side to side, just add a line like the previous
		\caption{picture(s) in a nutshell} (not necessary but it's useful) (won't compile if missing)
	\end{figure}
\end{comment}


%% pour insérer des liens
%\usepackage{hyperref}
%% usage :
\begin{comment}
	\href{the complete link}{the display text}
	or if you want to show the complete link, just use 
	\url{the complete link}
\end{comment}


%% to display code (for more informations, please see : https://texdoc.org/serve/listings.pdf/0)
\usepackage{listings}
\usepackage{xcolor}
%% usage :
\begin{comment}
	- listings package let you highly customize how you code will be displayed on the document
	- for simple display just do this :
	\lstdefinestyle{mystyle}{
		Your customizations
	}
	\begin{lstlisting}
		Your code
	\end{lstlisting}
\end{comment}
%% To display some code, I recommend this configuration
\lstdefinestyle{mystyle}{
	showspaces=false,
	showtabs=false,
	showstringspaces=false,
	numbers=left,
	language=Java,
	basicstyle=\footnotesize\ttfamily,
	frame=single,
	frameround=tttt
}
\lstset{style=mystyle}

%% to use maths and other sciences symbols (cf https://www.cmor-faculty.rice.edu/~heinken/latex/symbols.pdf) (
%\usepackage{amssymb,amsmath,amsfonts,extarrows}
%% usage : just type the backslash chararacter and the name of the symbol you want, depending of document on the the previous link

%% Have to find the goal of it
%\usepackage{soul}
%\let\oldemptyset\emptyset
%\usepackage[T1]{fontenc}

%% won't compile if at least one of these three next line is deleted
%% but you can add nothing between the brackets to leave it blank
\author{}
\date{}
\title{COMP 2522 - Winter 2024 - Quizz 6}

\begin{document}
\maketitle
%%\newpage
%% not necessary
%% compile twice the first time to display table of content
%%\tableofcontents
%%\newpage
Note : I put one page per question but the answer can be very short for some of them.
\begin{enumerate}
	\item There is an error (as shown in) in the code below. Why is there an error? What is the solution? (2 marks)
	\begin{lstlisting}
class Animal{
	public void draw(){}
}

class Dog extends Animal{}

public class Test{
	public void draw(List<Animal> list){
		for(Animal a : list)
			a.draw();
	}
	
	public static void main(String[] args){
		Test t = new Test();
		List<Dog> dogs = new ArrayList<Dog>();
		t.draw(dogs); // Error
	}
}
	\end{lstlisting}
	\underline{Answer :}
	\newpage
	\item The function only works with actual Groceries and not its subclasses. The programmer would like it to work with Apple and Fruit and possibly other objects as well as Groceries. Change the code so it will work and explain why ot will work. (2 marks)
	\begin{lstlisting}
class Groceries{}
class Fruit extends Groceries{}
class Apple extends Fruit{}

public void bagIt(List<Groceries> items, Groceries g){
	items.add(g);
}
	\end{lstlisting}
	\underline{Answer :}
	\newpage
	\item Which one of the following is NOT a subclass of Collection \textless String\textgreater? (1 mark)
	\begin{enumerate}
		\item Payload \textless String,String\textgreater
		\item Payload \textless Integer,Integer\textgreater
		\item Payload \textless String,Integer\textgreater
	\end{enumerate}
	\begin{lstlisting}
class List<T> extends Collection<T>{}
class ArrayList<T> extends List<T>{}
class Payload<X,T> extends ArrayList<T>{}
	\end{lstlisting}
	\item The code below does not compile. Why is there a problem? Fix the code so it will work. (2 marks)
	\begin{lstlisting}
void swapFirst(List<? extends Number> l1, List<? extends Number> l2){
	Number temp = l1.get(0);
	l1.set(0,l2.get(0));
	l2.set(0,temp);
}
	\end{lstlisting}
	\underline{Answer :}
\end{enumerate}
\end{document}
