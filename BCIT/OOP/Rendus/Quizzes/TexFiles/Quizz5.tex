\documentclass{article}
%% to use a package, just delete the % character to enable the package in this document, only on the lines where there is a single % character (otherwise it won't compile)
%% to understand the utility of the package, please read the comment between \begin{comment} and \end{comment} right below the package you need
\usepackage[a4paper, total={6in, 8in}]{geometry} % for european user

%% for multiple lines comment in LaTeX
\usepackage{verbatim} % (won't compile if this line is deleted)
%% usage :
\begin{comment}
	Your multiple lines comment
\end{comment}


%% to include images
%\usepackage{graphicx}
%% usage :
\begin{comment}
	\begin{figure}
		\centering (not necessary but it's prettier)
		\includegraphics[scale=(1 by default)]{path to the image/imageName.extension}
		\if you want to put another image side to side, just add a line like the previous
		\caption{picture(s) in a nutshell} (not necessary but it's useful) (won't compile if missing)
	\end{figure}
\end{comment}


%% pour insérer des liens
%\usepackage{hyperref}
%% usage :
\begin{comment}
	\href{the complete link}{the display text}
	or if you want to show the complete link, just use 
	\url{the complete link}
\end{comment}


%% to display code (for more informations, please see : https://texdoc.org/serve/listings.pdf/0)
\usepackage{listings}
\usepackage{xcolor}
%% usage :
\begin{comment}
	- listings package let you highly customize how you code will be displayed on the document
	- for simple display just do this :
	\lstdefinestyle{mystyle}{
		Your customizations
	}
	\begin{lstlisting}
		Your code
	\end{lstlisting}
\end{comment}
%% To display some code, I recommend this configuration
\lstdefinestyle{mystyle}{
	showspaces=false,
	showtabs=false,
	showstringspaces=false,
	numbers=left,
	language=Java,
	basicstyle=\footnotesize\ttfamily,
	frame=single,
	frameround=tttt
}
\lstset{style=mystyle}

%% to use maths and other sciences symbols (cf https://www.cmor-faculty.rice.edu/~heinken/latex/symbols.pdf) (
%\usepackage{amssymb,amsmath,amsfonts,extarrows}
%% usage : just type the backslash chararacter and the name of the symbol you want, depending of document on the the previous link

%% Have to find the goal of it
%\usepackage{soul}
%\let\oldemptyset\emptyset
%\usepackage[T1]{fontenc}

%% won't compile if at least one of these three next line is deleted
%% but you can add nothing between the brackets to leave it blank
\author{}
\date{}
\title{COMP 2522 - Winter 2024 - Quizz 5}

\begin{document}
\maketitle
%%\newpage
%% not necessary
%% compile twice the first time to display table of content
%%\tableofcontents
%%\newpage
Note : I put one page per question but the answer can be very short for some of them.
\begin{enumerate}
	\item Using the below explain why using Generics is beneficial. Why does it compile but not run? (2 marks)
	\begin{lstlisting}
String[] data = new String[1];
Object[] ptr;
ptr = data;
ptr[0] = 15;
	\end{lstlisting}
	\underline{Answer :}
	\newpage
	\item Demonstrate how you can use Generics to improve the below copy() method and WHY it is better than the original method. (2 marks)
	\begin{lstlisting}
void copy(Object[] a, Object[] b){
	System.array.copy(a,0,b,b.length); // assumes array b[] is large enough
}
	\end{lstlisting}
	\underline{Answer :}
	\newpage
	\item Write the function below as a generic function. (4 marks)
	\begin{lstlisting}
Interface Comparable<X>{
	int compareTo(X x);
}

public static int min(int[] data){
	int result = data[0];
	for(int value : data){
		if(result>value)
			result = value
	}
	return res;
}
	\end{lstlisting}
	\underline{Answer :}
\end{enumerate}
\end{document}
