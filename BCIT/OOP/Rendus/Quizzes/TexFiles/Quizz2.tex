\documentclass{article}
%% to use a package, just delete the % character to enable the package in this document, only on the lines where there is a single % character (otherwise it won't compile)
%% to understand the utility of the package, please read the comment between \begin{comment} and \end{comment} right below the package you need
\usepackage[a4paper, total={6in, 8in}]{geometry} % for european user

%% for multiple lines comment in LaTeX
\usepackage{verbatim} % (won't compile if this line is deleted)
%% usage :
\begin{comment}
	Your multiple lines comment
\end{comment}


%% to include images
%\usepackage{graphicx}
%% usage :
\begin{comment}
	\begin{figure}
		\centering (not necessary but it's prettier)
		\includegraphics[scale=(1 by default)]{path to the image/imageName.extension}
		\if you want to put another image side to side, just add a line like the previous
		\caption{picture(s) in a nutshell} (not necessary but it's useful) (won't compile if missing)
	\end{figure}
\end{comment}


%% pour insérer des liens
%\usepackage{hyperref}
%% usage :
\begin{comment}
	\href{the complete link}{the display text}
	or if you want to show the complete link, just use 
	\url{the complete link}
\end{comment}


%% to display code (for more informations, please see : https://texdoc.org/serve/listings.pdf/0)
\usepackage{listings}
\usepackage{xcolor}
%% usage :
\begin{comment}
	- listings package let you highly customize how you code will be displayed on the document
	- for simple display just do this :
	\lstdefinestyle{mystyle}{
		Your customizations
	}
	\begin{lstlisting}[language=<the language you'll use>]
		Your code
	\end{lstlisting}
\end{comment}
\lstdefinestyle{mystyle}{
	showspaces=false,
	showtabs=false,
	showstringspaces=false,
	numbers=left,
	language=Java,
	basicstyle=\footnotesize\ttfamily,
	frame=single,
	frameround=tttt
}
\lstset{style=mystyle}

%% to use maths and other sciences symbols (cf https://www.cmor-faculty.rice.edu/~heinken/latex/symbols.pdf) (
%\usepackage{amssymb,amsmath,amsfonts,extarrows}
%% usage : just type the backslash chararacter and the name of the symbol you want, depending of document on the the previous link

%% Have to find the goal of it
\usepackage{soul}
\let\oldemptyset\emptyset
\usepackage[T1]{fontenc}

%% won't compile if at least one of these three next line is deleted
%% but you can add nothing between the brackets to leave it blank
\author{}
\date{}
\title{COMP 2522 - Winter 2024 - Quizz 2}

\begin{document}
\maketitle
%%\newpage
%% not necessary
%% compile twice the first time to display table of content
%%\tableofcontents
%%\newpage

Note : I put one page per question but the answer can be very short for some of them.

\begin{enumerate}
	\item Is there an error at the line 6 ? Why or why not ? (2 marks)
	\begin{lstlisting}
class Foo{
	private Bar b;
	private int x;
	class Bar{
		public void setX(int value){
			x = value; // Error here
		}
	}
}
	\end{lstlisting}
	\underline{Answer :}
	\newpage
	\item Examine the main() method below to create a Bar object from the definition provided in Q1 above. The compiler say the line in main is in error. Why ? How do you fix it ? (2 marks)
	\begin{lstlisting}
public static void main(String[] args){
	Foo.bar bar = new Bar(); // Error
}
	\end{lstlisting}
	\underline{Answer :}
	\newpage
	\item Is there an error in the code below ? Why or why not ? (2 marks)
	\begin{lstlisting}
interface A{
	void foo();
}

class B implements A{
	public void foo(){}
}

class Main{
	public static void main(String[] args){
		A a = new B(); // Error ?	
	}
}
	\end{lstlisting}
	\underline{Answer :}
	\newpage
	\item Using the below interface definitions and Q3, is there any conflict in class J ? (2 marks)
	\begin{lstlisting}
interface X{
	void foo();
}

abstract class J implements A,X {}
	\end{lstlisting}
	\underline{Answer :}
	\newpage
	\item Modify the classes Animal, Dog and Cat as necessary and define CatDog to have Play functions as desired. Note Cat is-NOT a Dog is-NOT a Cat. Hint : Java does not allow multiple inheritance however it gives you something that allows a class to be many different types! Use that ability ! Yu cannot chnage class Play in any way. (3 marks)
	\begin{lstlisting}
class Animal{
	public void draw(){
		System.out.println("animal");
	}
}

class Dog{
	public void draw(){
		System.out.println("dog");
	}
	public void bark(){
		System.out.println("woof");
	}
}

class Cat{
	public void draw(){
		System.out.println("cat");
	}
	public void purr(){
		System.out.println("prrr");
	}
}

class CatDog{}

// No changes needed below this line
public class Play{
	public void guard(Dog d){
		d.bark();
	}
	public void pet(Cat c){
		c.purr();
	}
	public void make(Animal a){
		a.draw();
	}
	
	public static void main(String[] args){
		CatDog a = new CatDog();
		Play p = new Play();
		p.guard(a);
		p.pet(a);
		p.make(a);
	}
}

// Desired output :
// woof
// prrr
// catdog
	\end{lstlisting}
\end{enumerate}
\newpage
\underline{Answer :}
\end{document}
