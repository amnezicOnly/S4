\documentclass{article}
%% to use a package, just delete the % character to enable the package in this document, only on the lines where there is a single % character (otherwise it won't compile)
%% to understand the utility of the package, please read the comment between \begin{comment} and \end{comment} right below the package you need
\usepackage[a4paper, total={6in, 8in}]{geometry} % for european user

%% for multiple lines comment in LaTeX
\usepackage{verbatim} % (won't compile if this line is deleted)
%% usage :
\begin{comment}
	Your multiple lines comment
\end{comment}


%% to include images
%\usepackage{graphicx}
%% usage :
\begin{comment}
	\begin{figure}
		\centering (not necessary but it's prettier)
		\includegraphics[scale=(1 by default)]{path to the image/imageName.extension}
		\if you want to put another image side to side, just add a line like the previous
		\caption{picture(s) in a nutshell} (not necessary but it's useful) (won't compile if missing)
	\end{figure}
\end{comment}


%% pour insérer des liens
%\usepackage{hyperref}
%% usage :
\begin{comment}
	\href{the complete link}{the display text}
	or if you want to show the complete link, just use 
	\url{the complete link}
\end{comment}


%% to display code (for more informations, please see : https://texdoc.org/serve/listings.pdf/0)
\usepackage{listings}
\usepackage{xcolor}
%% usage :
\begin{comment}
	- listings package let you highly customize how you code will be displayed on the document
	- for simple display just do this :
	\lstdefinestyle{mystyle}{
		Your customizations
	}
	\begin{lstlisting}
		Your code
	\end{lstlisting}
\end{comment}
%% To display some code, I recommend this configuration

\lstdefinestyle{mystyle}{
	showspaces=false,
	showtabs=false,
	showstringspaces=false,
	numbers=left,
	language=Java,
	basicstyle=\footnotesize\ttfamily,
	frame=single,
	frameround=tttt
}
\lstset{style=mystyle}

%% to use maths and other sciences symbols (cf https://www.cmor-faculty.rice.edu/~heinken/latex/symbols.pdf) (
%\usepackage{amssymb,amsmath,amsfonts,extarrows}
%% usage : just type the backslash chararacter and the name of the symbol you want, depending of document on the the previous link

%% Have to find the goal of it
%\usepackage{soul}
%\let\oldemptyset\emptyset
%\usepackage[T1]{fontenc}

%% won't compile if at least one of these three next line is deleted
%% but you can add nothing between the brackets to leave it blank
\author{}
\date{}
\title{COMP 2522 - Winter 2024 - Quizz 3}

\begin{document}
\maketitle
%%\newpage
%% not necessary
%% compile twice the first time to display table of content
%%\tableofcontents
%%\newpage
Note : I put one page per question but the answer can be very short for some of them.
\begin{enumerate}
	\item The code below has an error. Why is there an error ? Fix the error and then give the output. (4 marks)
	\begin{lstlisting}
class Structure{
	public Structure(){
		System.out.println("structure");
	}
}

class Building extends Structure{
	int size;
	public Building(int s){
		size = s;
		System.out.println("building");
	}
}

class House extends Building{
	public House(){
		System.out.println("house");
	}
}

public class Construction{
	public static void main(String[] args){
		Structure s = new House();
	}
}
	\end{lstlisting}
	\underline{Answer :}
	\newpage
	\item Give the output for the program below. (4 marks)
	\begin{lstlisting}
class I{
	int i = 9;
	int j;
	I(){
		prt("i = "+i+", j = "+j+);
		j = 39;
	}
	
	static int x1 = prt("static I.x1 initialized");
	
	static int prt(String s){
		System.out.println(s);
		return 47;
	}
}

public class B extends I{
	int k = prt("B.k initialized");
	B(){
		prt("k = "+k);
		prt("j = "+j);
	}
	static int x2 = prt("B.x2 initialized");
	
	public static void main(String[] args){
		System.out.println("Started program");
		B b;
	}
}
	\end{lstlisting}
	\underline{Answer :}
	\newpage
	\item Give the output for the program below. Note the slight change from the previous question) (2 marks)
	\begin{lstlisting}
class I{
	int i = 9;
	int j;
	I(){
		prt("i = "+i+", j = "+j+);
		j = 39;
	}
	
	static int x1 = prt("static I.x1 initialized");
	
	static int prt(String s){
		System.out.println(s);
		return 47;
	}
}

public class B extends I{
	int k = prt("B.k initialized");
	B(){
		prt("k = "+k);
		prt("j = "+j);
	}
	static int x2 = prt("B.x2 initialized");
	
	public static void main(String[] args){
		new B();
	}
}
	\end{lstlisting}
	\underline{Answer :}
\end{enumerate}


\end{document}
