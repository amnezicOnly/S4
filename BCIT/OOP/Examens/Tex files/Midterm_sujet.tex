\documentclass{article}
%% to use a package, just delete the % character to enable the package in this document, only on the lines where there is a single % character (otherwise it won't compile)
%% to understand the utility of the package, please read the comment between \begin{comment} and \end{comment} right below the package you need
\usepackage[a4paper, total={6in, 8in}]{geometry} % for european user

%% for multiple lines comment in LaTeX
\usepackage{verbatim} % (won't compile if this line is deleted)
%% usage :
\begin{comment}
	Your multiple lines comment
\end{comment}


%% to include images
%\usepackage{graphicx}
%% usage :
\begin{comment}
	\begin{figure}
		\centering (not necessary but it's prettier)
		\includegraphics[scale=(1 by default)]{path to the image/imageName.extension}
		\if you want to put another image side to side, just add a line like the previous
		\caption{picture(s) in a nutshell} (not necessary but it's useful) (won't compile if missing)
	\end{figure}
\end{comment}


%% pour insérer des liens
%\usepackage{hyperref}
%% usage :
\begin{comment}
	\href{the complete link}{the display text}
	or if you want to show the complete link, just use 
	\url{the complete link}
\end{comment}


%% to display code (for more informations, please see : https://texdoc.org/serve/listings.pdf/0)
\usepackage{listings}
\usepackage{xcolor}
%% usage :
\begin{comment}
	- listings package let you highly customize how you code will be displayed on the document
	- for simple display just do this :
	\lstdefinestyle{mystyle}{
		Your customizations
	}
	\begin{lstlisting}[language=<the language you'll use>]
		Your code
	\end{lstlisting}
\end{comment}
\lstdefinestyle{mystyle}{
	showspaces=false,
	showtabs=false,
	showstringspaces=false,
	breaklines=true,
	numbers=left,
	language=Java,
	basicstyle=\footnotesize\ttfamily,
	frame=single,
	frameround=tttt
}
\lstset{style=mystyle}

%% to use maths and other sciences symbols (cf https://www.cmor-faculty.rice.edu/~heinken/latex/symbols.pdf) (
%\usepackage{amssymb,amsmath,amsfonts,extarrows}
%% usage : just type the backslash chararacter and the name of the symbol you want, depending of document on the the previous link

%% usage : to continue the same index counter for several splitted lists
\usepackage{enumitem}

%% Have to find the goal of it
%\usepackage{soul}
%\let\oldemptyset\emptyset
%\usepackage[T1]{fontenc}

%% won't compile if at least one of these three next line is deleted
%% but you can add nothing between the brackets to leave it blank
\author{}
\date{February 29th 2024}
\title{COMP 2522 : Midterm Exam (OOP using Java)}

\begin{document}
\maketitle
\begin{enumerate}
	\item The exam is \underline{closed book}. You may not use notes, texts, manuals during exam.
	\item Cheating will be results in disqualification from the exam and possibly expulsion from the course.
	\item No talking during exam. Please raise your hand if you have a question and wait for an instructor.
	\item There are 16 pages. Make sure you have all the pages \textbf{BEFORE} you start the exam.
	\item There are 4 parts to the exam. Section A is a multiple choice (8 marks); Section B is Give the output using multiple choices (16 marks); Section C is Find the error (4 marks); Section D is coding (6 marks). Max 34 marks. Make sure your FULL NAME and \textbf{SET} is on the exam.
	\item LISTEN TO THE INSTRUCTOR FOR FURTHER INSTRUCTIONS AND POSSIBLE CHANGES DURING THE EXAM. ALSO CHECK THE PROJECTOE DISPLAY OR WHITEBOARD.
	\item Time limit : \textbf{60 minutes}.
	\item Good luck !
\end{enumerate}
\newpage
%% not necessary
%% compile twice the first time to display table of content
%%\tableofcontents
%%\newpage

\section{Part A : Multiple choice (8 marks). Circle the best answer. 1 mark each}
\begin{enumerate}
	\item Interfaces allows for :
	\begin{enumerate}
		\item Multiple type matching
		\item Multiple method declaration without conflict
		\item Constants to be defined
		\item all of the above
		\item none of the above
	\end{enumerate}
	\begin{lstlisting} 
public class Head{
	Brain brain;
	private class Brain{}
			
	public static void main(String[] args){
		Head h = new Head();
	}
}			
	\end{lstlisting}
	\item The code to instantiate a Brain in method main() is :
	\begin{enumerate}
		\item Brain brain = new Brain();
		\item h.brain = new Brain();
		\item h.brain = h.new Brain();
		\item Head.Brain brain = new Brain();
		\item cannot be made, there is no constructor for a Brain object
	\end{enumerate}
	\newpage
	\begin{lstlisting} 
package midterm;
public class A{
	private void snafu(){}
	void foo(){}
	protected void bar(){}
}
	\end{lstlisting}
	\hrulefill
	\hrulefill
	\begin{lstlisting} 
package finalexam;
import midterm.A;
public class B extends A{}
	\end{lstlisting}
	\hrulefill
	\hrulefill
	\begin{lstlisting} 
package midterm;
public class C{
	A a = new A();
}
	\end{lstlisting}
	\item Using the above declarations, what methods can class B access from its parents :
	\begin{enumerate}
		\item foo();
		\item bar();
		\item snafu();
		\item foo(); and bar();
		\item foo(); bar(); and snafu();
	\end{enumerate}
	\item Using the above declarations, what methods can be accessed usign reference "\textbf{a}" :
	\begin{enumerate}
		\item a.foo();
		\item a.bar();
		\item a.snafu();
		\item a.foo(); and a.bar();
		\item a.foo(); , a.bar(); and a.snafu();
	\end{enumerate}
	\item Using the above declaration, in class B what \textbf{modifications} are allowed to the access modifiers to EXPLICITLY OVERRIDE methods from A? \textbf{Assume class B definition was moved to the same package "midterm" as class A.}
	\begin{enumerate}
		\item void foo(){} changed to protected void foo(){}, protected void bar(){} changed to public void bar(){}
		\item void foo(){} changed to private void foo(){}, protected void bar(){} changed to void bar(){}
		\item private snafu(){} changed to public void snafu(){}
		\item all of the above are valid
		\item none of the above are valid
	\end{enumerate}
	\begin{lstlisting} 
String[] names = {"hello","goodbye"};
Object[] ptr = names;
ptr[1] = 12.5;
	\end{lstlisting}
	\item The above is :
	\begin{enumerate}
		\item compile time error line 2
		\item runtime error line 2
		\item compile time error line 3
		\item runtime error line 3
		\item allowed because the two arrays are related through inheritance
	\end{enumerate}
	\begin{lstlisting} 
interface Transaction{
	public void run(){}
}

class BankTransaction implements Transaction{
	public void run(){}
}

class Test{
	public static void task(ArrayList<Transaction> list){
		for(Transaction trans : list)
			trans.run();
	}
	
	public static void main(String[] args){
		ArrayList<BankTransaction> t = new ArrayList<>();
		task(t);
	}
}
	\end{lstlisting}
	\item The above code :
	\begin{enumerate}
		\item will NOT compile, safest fix is task(ArrayList \textless X\textgreater list)
		\item will NOT compile, safest fix is task(ArrayList \textless ? extends Transaction\textgreater list)
		\item will NOT compile, safest fix is task(ArrayList list)
		\item will NOT compile, safest fix is task(ArrayList \textless X implements Transaction\textgreater list)
		\item will compile and run as expected
	\end{enumerate}
	\item Benefit(s) of using Generics are :
	\begin{enumerate}
		\item casting not required on returned values
		\item strong type checking during compile time rather than runtime
		\item methods can be used with any type
		\item all of the above
		\item none of the above
	\end{enumerate}
\end{enumerate} 

\section{Part B : Give the output. All the code here compiles and runs (16 marks)}
\begin{lstlisting}
class Tansaction{
	public void amount(){
		System.out.println("amount");
	}
	
	public Transaction(){
		System.out.println("Transaction created");
		amount();
	}
}

class Debit extends Transaction{
	public void amount(){
		System.out.println("debit amount");
	}
	
	public Debit(){
		System.out.println("Debit created")
		amount();
	}
}

public class Withdrawal extends Debit{
	int fee = 2;
	public void amount(){
		System.out.println("withdrawal and fee "+fee);
	}
	
	public Withdrawal(int n){
		System.out.println("Withdrawal created with a fee of " + fee);
		fee = n;
		System.out.println("fee="+fee);
	}
	
	public static void main(String[] args){
		Debit d = new Withdrawal(4);
	}
}
\end{lstlisting}
\begin{enumerate}[resume]
	\item Give the \textbf{output} for the above code : (2 mark)
	\begin{enumerate}
		\item Debit created, debit amount, Transaction created with a fee of 2, fee=4
		\item Transaction created, amount, Debit created, debit amount, Transaction created with a fee of 2, fee=4
		\item Transaction created, withdrawal and fee 2, Debit created, withdrawal and fee 2, Withdrawal created with a fee of 2, fee=4
		\item Transaction created, amount, debit amount, withdrawal and fee of 2, Debit created, debit amount, withdrawal and fee 2, Withdrawal created with fee of 2, fee=4
		\item Transaction created, withdrawal and fee 0, Debit created, withdrawal and fee 0, Withdrawal created with a fee of 2, fee=4
	\end{enumerate}
	\begin{lstlisting}
Class Cycle{
	int numWheels = 1;
	int wheelsize;
	Cycle(){
		output("number of wheels = " + numWheels + ", wheel size = "+ wheelsize);
		wheelsize = 24;
	}
	
	static int age = output("Cycle.age in years initialized");
	
	static int output(String s){
		System.out.println(s);
		return 2;
	}
}

class MountainBike extends Cycle{
	int numWheels = Cycle.output("MountainBike.numWheels initialized");
	MountainBike(){
		Cycle.output("snumWheels = " + numWheels);
		Cycle.output("wheel size = " + wheelsize);
	}
	
	static int hydrolicBrakes = Cycle.output("MountainBike.hydrolicBrakes initialized");
}

public class Trail{
	public static void main(String[] args){
		System.out.println("Started program");
		MountainBike m = new MountainBike();
	}
}
\end{lstlisting}
	\item Give the \textbf{exact output} for the above class : (2 marks)
	\begin{enumerate}
		\item Started program, MountainBike.hydrolicBrakes initialized, MountainBike.numWheels initialized, number of whells = 1, wheel size = 0, Cycle.age in years initialized, snumWheels = 1, wheel size = 24
		\item Started program, Cycle.age in years initialized, MountainBike.hydrolicBrakes initialized, number of wheels = 1, wheel size = 0, snumWheels = 1, wheel size = 24
		\item Cycle.age in years initialized, MountainBike.hydrolicBrakes initialized, Started program, number of wheels = 0, wheel size = 0, MountainBike.numWheels initialized, snumWheels = 1, wheel size = 24
		\item Cycle.age in years initialized, MountainBike.hydrolicBrakes initialized, Started program, number of wheels = 1, wheel size = 0, snumWheels = 1, wheel size = 24
		\item Started program, Cycle.age in years initialized, MountainBike.hydrolicBrakes initialized, number of wheels = 1, wheel size = 0, MountainBike.numWheels initialized, snumWheels = 2, wheel size = 24
	\end{enumerate}
	\newpage
	\begin{lstlisting}
public class Swapper{
	public static <T> void swap(T a, T b){
		T temp = a;
		a = b;
		b = temp;
	}
	
	public static <T> T swap(T a, T b, T c){
		T temp = a;
		a = b;
		b = c;
		c = temp;
		return c;
	}
	
	public static <T> void swap(T[] pool, int x, int y){
		T temp = pool[x];
		pool[x] = pool[y];
		pool[y] = temp;
	}
	
	public static void main(String[] args){
		String a = "hello";
		String b = "goodbye";
		String c = "fubar";
		Integer[] collection = {1,2,3,4,5};
		swap(a,b);
		System.out.println(a);
		a = "hello";
		b = "goodbye";
		c = "fubar";
		c = (String)swap(a,b,c);
		System.out.println(c);
		swap(collection,2,3);
		System.out.println(collection[3]);
		
	}
}
	\end{lstlisting}
	\item Give the \textbf{exact output} for the above code : (2 marks)
	\begin{enumerate}
		\item hello, hello, 3
		\item hello, fubar, 4
		\item goodbye, hello, 2
		\item hello, hello, 2
		\item hello, goodbye, 2
	\end{enumerate}
	\newpage
	\begin{lstlisting}
interface Car{
	void start();
}

class Ford implements Car{
	public void start(){
		System.out.println("F");
	}
}

class GM extends Ford implements Car{
	public void start(){
		System.out.println("G");
	}
}

class Chrystler extends Ford implements Car{
	public void start(){
		System.out.println("C");
	}
}

class Ram extends GM{
	public void start(){
		super.start();
		System.out.println("R");
	}
}

public class ParkingLot{
	public static void main(String[] args){
		Car x;
		x = new Ford();
		x.start();
		x = new GM();
		x.start();
		x = new Chrystler();
		x.start();
		x = new Ram();
		x.start()
	}
}
	\end{lstlisting}
	\item Give the output for the above code : (2 marks)
	\begin{enumerate}
		\item F, G, C, R
		\item F, F, G, F, C, F, R
		\item F, F, F, G, F, C, G, F
		\item F, G, C, G, R
		\item syntax error, code line "Car x;" is illegal
	\end{enumerate}
	\newpage
	\begin{lstlisting}
public class Experiment{
	public static void heat() throws Exception{
		if(Math.random(2)==1)
			throw new Exception();
	}
	
	public static void test() throws Exception{
		try{
			heat();
			System.out.println("matter heated");
		} catch (Exception e) {
			System.out.println("exception occurred");
		} finally {
			System.out.println("clean up site");
		}
		System.out.println("end of test");
	}
	
	public staic void main(String[] args){
		Experiment e = new Experiment();
		e.test();
	}
}
	\end{lstlisting}
	\item Give the output of the above code \textbf{IF method heat() \underline{throws} an exception}: (2 marks)
	\begin{enumerate}
		\item matter heated, exception occurded, clean up site
		\item exception occurred, matter heated, clean up site, end of test
		\item exception occurred, clean up site
		\item exception occurred, end of test
		\item exception occurred, clean up site, end of test
	\end{enumerate}
	\item Give the output of the above code \textbf{IF method heat() \underline{DOES NOT} throw an exception}: (2 marks)
	\begin{enumerate}
		\item matter heated, clean up site
		\item matter heated, clean up site, end of test
		\item matter heated, end of test
		\item end of test
		\item matter heated, end of test, clean up site
	\end{enumerate}
	\newpage
	\begin{lstlisting}
class WarpException extends Exception{}
class DissolveExcepion extends Exception{}

class Morpher{
	public static void warp() throws WarpException{
		throw new WarpException();
	}
	
	public static void dissolve() throws DissolveException{
		throw new DissolveException();
	}
	
	public static void main(String[] args){
		try{
			warp();
			System.out.println("warped");
		} finally {
			System.out.println("dissolve applied");
			dissolve();
		}
		System.out.println("finished");
	}
}
	\end{lstlisting}
	\item Give the output for the above code : (2 marks)
	\begin{enumerate}
		\item WarpException, warped, dissolve applied, DissolveException, finished
		\item dissolve applied, WarpException, DissolveException
		\item dissolve applied, DissolveException
		\item dissolve applied, finished
		\item warped, dissolve applied, finished, WarpException, DissolveException
	\end{enumerate}
	\newpage
	\begin{lstlisting} 
class Cup{
	Cup(int marker){
		System.out.println("Cup("+marker+")");
	}
	
	void f(int marker){
		System.out.println("f("+marker+")");
	}
}

class Cups{
	static Cup c1 = new Cup(1);
	static Cup c2 = new Cup(2);
	
	Cups(){
		System.out.println("Cups()");
	}
}

public class ExplicitStatic{
	public static void main(String[] args){
		System.out.println("Inside main()");
		new Cups();
		Cups.c1.f(99);
	}
	static Cups x = new Cups();
	static Cups y = new Cups();
}

	\end{lstlisting}
	\item Give the output of the above code : (2 marks)
	\begin{enumerate}
		\item Inside main(), Cups(), f(99)
		\item Cups(), Cups(), Cup(1), Cup(2), Inside main(), Cups(), f(99)
		\item Inside main(), Cups(1), Cup(2), Cups(), Cups, Cups(), f(99)
		\item Cups(), Cup(1), Cup(2), Cups(), Inside main(), Cups(), f(99)
		\item Cup(1), Cup(2), Cups(), Cups(), Inside main(), Cups(), f(99)
	\end{enumerate}
\end{enumerate}
\newpage
\section{Part C : Error code. Code here has one error. Explain why the code is in error  and provide a fix so that the intent of the author remains. (4 marks)}
\begin{lstlisting}
class Container{	// DO NOT CHANGE THIS CLASS
	protected String id;
	public Container(String n){
		id = n;	
	}
}

public class ShippingContainer extends Container{
	int size;
	public ShippingContainer(String n){	// COMPILE ERROR HERE
		id = n;
	}
	public void setSize(int s){
		size = s;
	}
	public static void main(String[] args){	// DO NOT CHANGE MAIN
		ShippingContainer box = new ShippingContainer("456-123");
	}
}
\end{lstlisting}
\begin{enumerate}[resume]
	\item The above code doesn't \underline{\textbf{compile}}. Explain \textbf{why and fix the code} so it compiles and runs \textbf{as expected}. (2 marks)
	\newpage
	\begin{lstlisting}
public class Table{
	public static void addNumbers(List<Integer> list){
		for(int i=1; i<=10 ; i++)
			list.add(i);
	}
	
	public static void main(String[] args){
		List<Integer> test = newArrayList<>();
		addNumbers(test);
		List<Number> data = new ArrayList<>();
		addNumbers(data);	// ERROR
	}
}
	\end{lstlisting}
	\item The above code doesn't compile. Explain why and fix the code so it will compile and run as expected. (2 marks)
\end{enumerate}
\newpage
\section{Part D : Coding (6 marks)}
\begin{lstlisting}
class Agent{
	public void spy(){
		System.out.println("spy");
	}
}
class Gambler{
	public void gamble(){
		System.out.println("gamble");
	}
}
class Killer{
	public void kill(){
		System.out.println("bang");
	}
}
class JamesBond{}

// Do not change any of this code
public class Spectre{
	public void sneak(Agent a){
		a.spy();
		System.out.println("found secret");
	}
	public void cardGame(Gambler g){
		g.gamble();
		System.out.println("broke the bank");
	}
	public void getSpectre(Killer k){
		k.kill();
		System.out.println("dead");
	}
	
	public static void main(String[] args){
		Spectre g = new Spectre();
		JamesBond d = new JamesBond();
		g.sneak(b);
		g.cardGame(b);
		g.getSpectre(b);
		System.out.println("Crime does not pay");
	}
}
\end{lstlisting}
\begin{enumerate}[resume]
	\item Change the code above so that JamesBond can be passed to all the methods in class Spectre so crime will not win! \textbf{Note: Agent, Gambler, Killer are NOT similar types (MUTUALLY EXCLUSIVE!)}. Hint : a JamesBond is-a Agent, JamesBond is-a Gambler, JamesBond is-a Killer BUT \textbf{"Agent is-a Gambler" is NOT TRUE etc.} (2 marks)
	\newpage
	\underline{Answer :}
	\newpage
	\begin{lstlisting}
interface Iterator{
	boolean hasNext();
	int next();
}

public class List{
	int pos = 0;
	int[] data;
	
	class MyIterator implements Iterator{
		public int next(){
			return data[pos++];
		}
		public boolean hasNext(){
			return ((pos>0) && (pos<data.length));
		}
	}
}
	\end{lstlisting}
	\item Give the generic version of the code above. (3 marks)
\end{enumerate}

\end{document}
