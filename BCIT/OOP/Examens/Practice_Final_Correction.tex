\documentclass{article}
%% to use a package, just delete the % character to enable the package in this document, only on the lines where there is a single % character (otherwise it won't compile)
%% to understand the utility of the package, please read the comment between \begin{comment} and \end{comment} right below the package you need
\usepackage[a4paper, total={6in, 8in}]{geometry} % for european user

%% for multiple lines comment in LaTeX
\usepackage{verbatim} % (won't compile if this line is deleted)
%% usage :
\begin{comment}
	Your multiple lines comment
\end{comment}


%% to include images
%\usepackage{graphicx}
%% usage :
\begin{comment}
	\begin{figure}
		\centering (not necessary but it's prettier)
		\includegraphics[scale=(1 by default)]{path to the image/imageName.extension}
		\if you want to put another image side to side, just add a line like the previous
		\caption{picture(s) in a nutshell} (not necessary but it's useful) (won't compile if missing)
	\end{figure}
\end{comment}


%% pour insérer des liens
%\usepackage{hyperref}
%% usage :
\begin{comment}
	\href{the complete link}{the display text}
	or if you want to show the complete link, just use 
	\url{the complete link}
\end{comment}


%% to display code (for more informations, please see : https://texdoc.org/serve/listings.pdf/0)
\usepackage{listings}
\usepackage{xcolor}
%% usage :
\begin{comment}
	- listings package let you highly customize how you code will be displayed on the document
	- for simple display just do this :
	\lstdefinestyle{mystyle}{
		Your customizations
	}
	\begin{lstlisting}
		Your code
	\end{lstlisting}
\end{comment}
%% To display some code, I recommend this configuration
\lstdefinestyle{mystyle}{
	showspaces=false,
	showtabs=false,
	showstringspaces=false,
	numbers=left,
	language=Java,
	basicstyle=\footnotesize\ttfamily,
	frame=single,
	frameround=tttt
}
\lstset{style=mystyle}

%% to use maths and other sciences symbols (cf https://www.cmor-faculty.rice.edu/~heinken/latex/symbols.pdf) (
%\usepackage{amssymb,amsmath,amsfonts,extarrows}
%% usage : just type the backslash chararacter and the name of the symbol you want, depending of document on the the previous link

%% Have to find the goal of it
\usepackage{soul}
\let\oldemptyset\emptyset
\usepackage[T1]{fontenc}

%% won't compile if at least one of these three next line is deleted
%% but you can add nothing between the brackets to leave it blank
\author{}
\date{}
\title{Practice Final Correction}

\begin{document}
\maketitle
%%\newpage
%% not necessary
%% compile twice the first time to display table of content
%%\tableofcontents
%%\newpage
\begin{enumerate}
\begin{lstlisting}
class Race{
	public static final int MAX = 5;
 	public static final int MIN = 1;

 	private static int data=Race.MIN;
	public static int getData(){
		return data;
	}
	public static void incData() {
		if (data < Race.MAX)
			data++;
	}
	public static void foo() throws Exception{
		while (data < Race.MAX)
			incData();
	}
}

public class Car implements Runnable { 
	public void run(){
		Race.foo();
	}
	public static void main(String[] args) throws Exception{
  		Thread t1 = new Thread(new Car());
  		Thread t2 = new Thread(new Car());
  		t1.start();
  		t2.start();
  		t1.join();
  		t2.join();
  		System.out.println(Race.getData());
 	}
}
\end{lstlisting}
	\item In the above code, where might be race condition(s) occur? Speciafically which line(s) of the code ?
	\item Where and how should the race condition(s) be prevented?
	\item What is a deadlock? Can one occur in the above code ever? Why or why not?
\newpage
\begin{lstlisting}
class Examine{
 	static int[] data = new int[10];
 	public static void dance(int value){
  		int i=0;
  		while(i<data.length){
   			for(int j=0;j<100;j++);
   				data[i] = value;
  		}
 	}
 	static Runnable launch(int id){
  		return new Runnable(){
   			public void run(){
    				dance(id);
   		}
 	}
 	public static void main(String[] args){
  		Thread t1 = new Thread(launch(1));
  		Thread t2 = new Thread(launch(2));
  		t1.start();
  		t2.start();
  		t1.join();
  		t2.join();
 	}
}
\end{lstlisting}
	\item In the above code is there a race condition? If yes, what is the race condition and where exactly is it (demonstrate the race via code walk thru)? 
	\item Would the code when run cause an Exception? 
\newpage
\begin{lstlisting}
class Grain {
  	public String toString() { return "Grain"; }
}

class Wheat extends Grain {
  	public String toString() { return "Wheat"; }
}
class Mill {
  	Grain process() { return new Grain(); }
}
class WheatMill extends Mill {
  	Wheat process() { return new Wheat(); }
}
public class CovariantReturn {
  	public static void main(String[] args) {
    		Mill m = new Mill();
    		Grain g = m.process();
    		System.out.println(g);
    		m = new WheatMill();
    		g = m.process();
    		System.out.println(g);
  	}
}
\end{lstlisting}
	\item Is the method "process()" in WheatMill in error?
	\newpage
\begin{lstlisting}
class Egg2 {
  	protected class Yolk {   
    		public Yolk() { print("Egg2.Yolk()"); }       
    		public void f() { print("Egg2.Yolk.f()");}
  	}
  	private Yolk y = new Yolk();
  	public Egg2() { print("New Egg2()"); }
  	public void insertYolk(Yolk yy) { y = yy; }
  	public void g() { y.f(); }
} 

public class BigEgg2 extends Egg2 {
  	public class Yolk extends Egg2.Yolk {
    		public Yolk() { print("BigEgg2.Yolk()"); }
    		public void f() { print("BigEgg2.Yolk.f()"); }
  	}
  	public BigEgg2() { insertYolk(new Yolk()); }
  	public static void main(String[] args) {
    		Egg2 e2 = new BigEgg2();
    		e2.g();
  	}
}
\end{lstlisting}
	\item Give the output for the above:
	\newpage
	
\begin{lstlisting}
class Parcel4 {
  	private class PContents implements Contents {
    		private int i = 11;
    		public int value() { return i; }
  	}
  	protected class PDestination implements Destination {
    		private String label;
    		private PDestination(String whereTo) {
      			label = whereTo;
    		}
    		public String readLabel() { return label; }
  	}
  	public Destination destination(String s) {
    		return new PDestination(s);
  	}
  	public Contents contents() {
    		return new PContents();
    	}
}
public class TestParcel {
	public static void main(String[] args) {
		Parcel4 p = new Parcel4();
		Contents c = p.contents();
		Destination d = p.destination("Tasmania");
		Parcel4.PContents pc = p.new PContents();//ERROR
  	}
}
\end{lstlisting}
	\item The above class has an error on the line indicated. Explain why there is an error.\newpage
	
\begin{lstlisting}
public class Wrapping {
  	private int i;
  	public Wrapping(int x) { i = x; }
  	public int value() { return i; }
}

public class Parcel {
  	public Wrapping wrapping(int x) {
    	//return an anonymous inner class object of Wrapping type
   	//with overloaded method "public int value (){ return 47*i;}"
  	}
  	public static void main(String[] args) {
    		Parcel p = new Parcel();
    		Wrapping w = p.wrapping(10);
  	}
}
\end{lstlisting}
	\item In the above classes provide the missing code.\newline\newline\newline\newline\newline\newline
	\item Given an ArryList is-a List and a List is-a Collection is the following true? ArrayList<String> is-a Collection<String>\newline\newline
	\item Given interface FooBar<X,Y> extends Silly<X> which of the following ARE subtypes of Silly<String>?
	\begin{enumerate}
		\item FooBar<String, String>
		\item FooBar<String, Integer>
		\item FooBar<Integer, String>
		\item FooBar<String, Exception>
		\item FooBar<Integer, Integer>
		\item FooBar<Exception, Integer>
	\end{enumerate}
	\item Given
	\begin{quote}
		static \textless T\textgreater T pick(T a, T b){return b}
	\end{quote}
	Is the following an error or not? If an error, explain why, if not give the return type.
	\begin{quote}
		Collection c = pick(new Set\textless String\textgreater (), new Stack\textless String\textgreater());
	\end{quote}\newpage
	\item If the code below gives an error explain why, if not explain why.
\begin{lstlisting}
public static void addNumbers(List<? super Integer> list){
	for (int i=1; i<10; i++){
		list.add(i);
	}
}
// in other code:
addNumbers(new ArrayList<Number>());
\end{lstlisting}
	\item If the code below gives an error explain why, if not explain why.
\begin{lstlisting}
void swapFirst(List<? extends Number>listA, List<? extends Number> listB){
	Number temp = listA.get(0);
 	listA.set(0,listB.get(0));
	listB.set(0,temp);
}
\end{lstlisting}
\begin{lstlisting}
class Example{
 	public void open() throws FileNotFoundException{
  		System.out.println("attempting to open file");
  		throw new FileNotFoundException();
 	}
 	public void close() throws CloseException {
  		System.out.println("attempting to close file");
  		throw new CloseException();
 	}
 	public static void main(String[] args) throws Exception{
  		Example e = new Example();
  		try{ 
   			e.open();
   			System.out.println("after opening file");
  		}finally{ 
   			System.out.println("finally");
   			e.close();
   			System.out.println("after closing file");
  		}
  	System.out.println("end of program");
	}
}
\end{lstlisting}
	\item Give the output. State any exception(s) that are displayed on exit.
	\newpage
\begin{lstlisting}
class LanguageException extends Exception{}
	class JavaException extends LanguageException{}
	public class Test {
 		public void a() throws LanguageException{
  			throw new LanguageException();
 		}
 	public void b() throws JavaException{
  		throw new JavaException();
 	}
 	public static void main(String[] args){
  		Test t = new Test(); 
  		try{
   			t.a();
   			t.b();
  		}
  		catch(LanguageException l){}
  		catch(JavaException j){}
  		System.out.println("finished main");
 	}
}
\end{lstlisting}
	\item Give the output. State any exception(s) that are displayed on exit.
\begin{lstlisting}
public class Out {
 	int x;
 	static class In {
  		public void setX(int value){
   			x = value;
  		}
 	}
 	public static void main(String[] args){
  		//your code goes here
 	}
}
\end{lstlisting}
	\item What is the error in the above class? Why?
	\item Give the code to create an "In" object in main()
	\newpage
\begin{lstlisting}
class Cat {
 	Kitten k = new Kitten();
 	public Cat(){
  		System.out.println("cat");
 	}
 	class Kitten{
  		public Kitten(){
   			System.out.println("kitten");
  		}
 	}
}
public class Lion extends Cat {
 	public Lion(){
  		System.out.println("Lion");
 	}
 	class Kitten {
  		public Kitten(){
   			System.out.println("young Lion");
  		}
 	}
 	public static void main(String[] args){
  		new Lion();
 	}
}
\end{lstlisting}
	\item Give the output
\begin{lstlisting}
class Cat {
 	Kitten k;
 	public Cat(){
  		System.out.println("cat");
 	}
 	class Kitten{
  		public Kitten(){
   			System.out.println("kitten");
  		}
 	}
 	public void produce(Kitten kk){
   		k = kk;
 	}
}
public class Lion extends Cat {
 	public Lion(){
  		System.out.println("Lion");
  		produce(new Kitten());
 	}
 	class Kitten {
  		public Kitten(){
   			System.out.println("young Lion");
  		}
 	}
 	public static void main(String[] args){
  		new Lion();
 	}
}
\end{lstlisting}
	\item Give the output or if there Is an error, fix it and give the output.
\begin{lstlisting}
class A {
 	private int x;
 	public void setZ(int zz){
 		z = zz;
 	}
 	class B{
  		private int y;
  		class C{
   			private int z;
   			public void setX(int xx){
   				x = xx;
   			}
  		}
 	}
}
\end{lstlisting}
	\item Examine the code above. Is there an error? If so what is the error and why? If not, explain
\begin{lstlisting}
class A{
 	class B{
  		class C{}
 	}
	public static void main(String[] args){
 		//code
	}
}
\end{lstlisting}
	\item Give the code necessary to create a C object in main()
\begin{lstlisting}
class X {
 	int z = 5;
 	static class Y{
  		public int getZ(){return z;}
 	}
 	public static void main(String[] args){
 		//code
 	}
}
\end{lstlisting}
	\item What is the an error in the code above?
	\item If the error was removed in the above code, give the  code to create a Y object
	\newpage
\begin{lstlisting}
class X{
 	int y;
 	public void foo(){
  		for(int i=0; i<10; i++){
   			if (y < 5)
    				y++;
 		}
 	}
}
\end{lstlisting}
	\item For the above code give the places where a race condition exists.
\begin{lstlisting}
class A implements Runnable {
 	ReentrantLock lock = new ReentrantLock();
 	List s ;
 	public A(List store){
 		s = store;
 	}
 	public void run(){
 		for(int i=0;i<10; i++){
   			if(lock.trylock())
    				s.add(i);
    		}
 	}
}

class B implements Runnable{
 	ReentrantLock lock = new ReentrantLock();
 	List s;
 	public B(List store){
 		s = store;
 	}
 	public void run(){
  		for(int x = 20; x<30; x++){
   			if(lock.tryLock())
    				s.add(x);
    		}
 	}
}
public class Test {
 	public static void main(String[] args){
  		List store = new LinkedList();
  		A a = new A(store);
  		B b = new B(store);
  		new Thread(a).start();
  		new Thread(b).start();
  		Thread.sleep(2000);
 	}
}
\end{lstlisting}
	\item Does the above code protect the shared List? Why or why not?
	\newpage
\begin{lstlisting}
public class LTest {
 	public static void main(String[] args){
  		new Thread(new Runnable(){
     			public void run(){
      				System.out.println("hello world!");
     			}
     		}).start();
 
 	}
}
\end{lstlisting}
	\item Write the above using a Lambda expression\newpage
\begin{lstlisting}
public class Table{
 	enum TYPE {MULT, ADD}
 	public void display(int start, int end, TYPE t){
  		for (int i = start; i<end; i++){
   			for (int j = start; j <end; j++){
    				if (t == TYPE.MULT)
     					System.out.printf("%3d ",(i*j));
    				else 
     					System.out.printf("%3d ",(i+j));
   			}
   			System.out.println("");
  		}
 	}
 	public static void main(String[] args){
  		Table t = new Table();
  		t.display(1,10,TYPE.MULT);
 	}
}
\end{lstlisting}
	\item Modify to use a Lambda expression and give the expression to get the same output as the code above with your new display() method.
\end{enumerate}
\end{document}
