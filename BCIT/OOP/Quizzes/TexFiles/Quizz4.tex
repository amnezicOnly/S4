\documentclass{article}
%% to use a package, just delete the % character to enable the package in this document, only on the lines where there is a single % character (otherwise it won't compile)
%% to understand the utility of the package, please read the comment between \begin{comment} and \end{comment} right below the package you need
\usepackage[a4paper, total={6in, 8in}]{geometry} % for european user

%% for multiple lines comment in LaTeX
\usepackage{verbatim} % (won't compile if this line is deleted)
%% usage :
\begin{comment}
	Your multiple lines comment
\end{comment}


%% to include images
%\usepackage{graphicx}
%% usage :
\begin{comment}
	\begin{figure}
		\centering (not necessary but it's prettier)
		\includegraphics[scale=(1 by default)]{path to the image/imageName.extension}
		\if you want to put another image side to side, just add a line like the previous
		\caption{picture(s) in a nutshell} (not necessary but it's useful) (won't compile if missing)
	\end{figure}
\end{comment}


%% pour insérer des liens
%\usepackage{hyperref}
%% usage :
\begin{comment}
	\href{the complete link}{the display text}
	or if you want to show the complete link, just use 
	\url{the complete link}
\end{comment}


%% to display code (for more informations, please see : https://texdoc.org/serve/listings.pdf/0)
\usepackage{listings}
\usepackage{xcolor}
%% usage :
\begin{comment}
	- listings package let you highly customize how you code will be displayed on the document
	- for simple display just do this :
	\lstdefinestyle{mystyle}{
		Your customizations
	}
	\begin{lstlisting}
		Your code
	\end{lstlisting}
\end{comment}
%% To display some code, I recommend this configuration
\lstdefinestyle{mystyle}{
	showspaces=false,
	showtabs=false,
	showstringspaces=false,
	numbers=left,
	language=Java,
	basicstyle=\footnotesize\ttfamily,
	frame=single,
	frameround=tttt
}
\lstset{style=mystyle}

%% to use maths and other sciences symbols (cf https://www.cmor-faculty.rice.edu/~heinken/latex/symbols.pdf) (
%\usepackage{amssymb,amsmath,amsfonts,extarrows}
%% usage : just type the backslash chararacter and the name of the symbol you want, depending of document on the the previous link

%% Have to find the goal of it
%\usepackage{soul}
%\let\oldemptyset\emptyset
%\usepackage[T1]{fontenc}

%% won't compile if at least one of these three next line is deleted
%% but you can add nothing between the brackets to leave it blank
\author{}
\date{}
\title{COMP 2522 - Winter 2024 - Quizz 3}

\begin{document}
\maketitle
%%\newpage
%% not necessary
%% compile twice the first time to display table of content
%%\tableofcontents
%%\newpage
Note : I put one page per question but the answer can be very short for some of them.
\begin{enumerate}
	\item Give the output for the code below as is (note line 18 commented out) (2 marks)
	\begin{lstlisting}
public class CoolException{
	void f() throws Exception{
		throw new Exception();
	}
	
	void bar() throws Exception{
		throw new Exception();
	}
	
	void foo() throws Exception{
		System.out.println("starting foo()");
		try{
			System.out.println("in try before f()");
			f();
			System.out.println("in try after f()");
		} catch (Exception e) {
			System.out.println("exception occured");
			// bar();
		}
		System.out.println("end of foo");
	}
	
	public static void main(String[] args){
		CoolException c = new CoolException();
		try{
			c.foo();
		} catch (Exception e) {
			System.out.println("main caught exception");
		}
	}
}
	\end{lstlisting}
	\underline{Answer :}
	\newpage
	\item Give the output for the question 1 if the commented-out code was no longer commented out. (2 marks)\newline
	\underline{Answer :}
	\newpage
	\item Give the output of the code below. (4 marks)
	\begin{lstlisting}
class Shape{
	public void draw(){
		System.out.println("drawing shape");
	}
	public Shape(){
		System.out.println("creating shape");
		draw();
	}
}

class Rectangle extends Shape{
	int w = 9;
	int h = 15;
	public void draw(){
		System.out.println("drawing rectangle size = "+w+" "+h);
	}
	
	public Rectangle(int width, int height){
		System.out.println("creating shape");
		System.out.println("width = "+w);
		System.out.println("height = "+h);
		w = width;
		h = height;
	}
}

public class Drawing{
	public static void main(String[] args){
		Shape s = new Rectangle (100,200);
	}
}
	\end{lstlisting}
	\underline{Answer :}
\end{enumerate}

\end{document}
