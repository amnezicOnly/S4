\documentclass{article}
%% to use a package, just delete the % character to enable the package in this document, only on the lines where there is a single % character (otherwise it won't compile)
%% to understand the utility of the package, please read the comment between \begin{comment} and \end{comment} right below the package you need
\usepackage[a4paper, total={6in, 8in}]{geometry} % for european user

%% for multiple lines comment in LaTeX
\usepackage{verbatim} % (won't compile if this line is deleted)
%% usage :
\begin{comment}
	Your multiple lines comment
\end{comment}


%% to include images
%\usepackage{graphicx}
%% usage :
\begin{comment}
	\begin{figure}
		\centering (not necessary but it's prettier)
		\includegraphics[scale=(1 by default)]{path to the image/imageName.extension}
		\if you want to put another image side to side, just add a line like the previous
		\caption{picture(s) in a nutshell} (not necessary but it's useful) (won't compile if missing)
	\end{figure}
\end{comment}


%% pour insérer des liens
%\usepackage{hyperref}
%% usage :
\begin{comment}
	\href{the complete link}{the display text}
	or if you want to show the complete link, just use 
	\url{the complete link}
\end{comment}


%% to display code (for more informations, please see : https://texdoc.org/serve/listings.pdf/0)
\usepackage{listings}
\usepackage{xcolor}
%% usage :
\begin{comment}
	- listings package let you highly customize how you code will be displayed on the document
	- for simple display just do this :
	\lstdefinestyle{mystyle}{
		Your customizations
	}
	\begin{lstlisting}[language=<the language you'll use>]
		Your code
	\end{lstlisting}
\end{comment}
\lstdefinestyle{mystyle}{
	showspaces=false,
	showtabs=false,
	showstringspaces=false,
	numbers=left,
	language=Java,
	basicstyle=\footnotesize\ttfamily,
	frame=single,
	frameround=tttt
}
\lstset{style=mystyle}

%% to use maths and other sciences symbols (cf https://www.cmor-faculty.rice.edu/~heinken/latex/symbols.pdf) (
%\usepackage{amssymb,amsmath,amsfonts,extarrows}
%% usage : just type the backslash chararacter and the name of the symbol you want, depending of document on the the previous link

%% Have to find the goal of it
\usepackage{soul}
\let\oldemptyset\emptyset
\usepackage[T1]{fontenc}

%% won't compile if at least one of these three next line is deleted
%% but you can add nothing between the brackets to leave it blank
\author{}
\date{}
\title{COMP 2522 - Winter 2024 - Quizz 1}

\begin{document}
\maketitle
%%\newpage
%% not necessary
%% compile twice the first time to display table of content
%%\tableofcontents
%%\newpage

Note : I put one page per question but the answer can be very short for some of them.

\begin{enumerate}
	\item Give the answer to the lines indicated with comments (lines 19, 20 and 21). (3 marks)
	\begin{lstlisting}
class Movie{
	public void play(){
		System.out.println("playing movie");
	}
}

class DVD extends Movie{
	public void play(){
		System.out.println("playing DVD");
	}
	public void menu(){
		System.out.println("showing menu");
	}
}

public class Main{
	public static void main(String[] args){
		Movie m = new DVD();
		m.play(); // 1 : give the output or indicate an error and explain why
		DVD d = m; // 2 : Error ? If yes, fix if possible
		m = new DVD();
		m.menu(); // 3 : give the output or indicate an error and explain why
	}
}
	\end{lstlisting}
	\underline{Answer :}
	\newpage
	\item Give the output for the code below. Draw a picture if you want. (2 marks)
	\begin{lstlisting}
public class X{
	static void replace(String o){
		o = "Mystery";
	}
	
	static void swap(String[] a, int i, int j){
		String temp = a[i];
		a[i] = a[j];
		a[j] = temp;
	}
	
	public static void main(String[] args){
		String[] words = {"hello","goodbye","crazy"};
		replace(words[0]);
		System.out.println(words[0]);
		swap(words,0,2);
		System.out.println(words[0]);
	}
}
	\end{lstlisting}
	\underline{Answer :}
	\newpage
	\item Which function is in error in the code? Explain WHY (don't say it's a rule, explain WHY it makes sense that function is in error). (2 marks)
	\begin{lstlisting}
class Y{
	private static int x;
	private int y;
	
	public void setX(int a){
		x = a;
	}
	
	public void setY(int a){
		y = a;
	}
	
	public static void changeX(int a){
		x = a;
	}
	
	public static void change(int a){
		y = a;
	}
}
	\end{lstlisting}
	\underline{Answer :}
\end{enumerate}
\newpage
\end{document}
